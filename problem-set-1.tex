\documentclass[11pt]{scrartcl}
\usepackage{dominatrix}
\usepackage{solarized-light}
\lstset{
language=python
}

\renewcommand\thesection{Problem \arabic{section}}
\renewcommand\thesubsection{\arabic{section} (\alph{subsection})}
\renewcommand\thesubsubsection{(\roman{subsubsection})}
\DeclareMathOperator{\AU}{AU}
\DeclareMathOperator{\arcsec}{arcsec}
\DeclareMathOperator{\pc}{pc}
\DeclareMathOperator{\km}{km}
\DeclareMathOperator{\cm}{cm}
\DeclareMathOperator{\s}{s}

\newcommand\pow[2]{\ensuremath{#1 \times 10^{#2}}}

\title{Problem Set 1}
\subject{ASTR 1404 Stars, Galaxies, and Cosmology}
%\author{James H. Applgate}
\begin{document}
\maketitle

Recall the \textbf{Law of Skinny Triangles}

\[a = \theta \cdot D\]

where $\theta$ is in radians, $a$ and $D$ are in same units.

An alternative way to write this is in AU, arcsecs, and parsecs (pc):

\[a(\AU) = \theta (\arcsec) \cdot D (\pc)\]

where $1\AU = \pow{1.5}{13}$ and $1\pc = \pow{3.09}{18}$

\section{}

Compute the number of

\begin{itemize}
\item degrees in a radian
\item arcminutes in a radian
\item arcseconds in a radian
\item radians in a degree
\item radians in a arcminute
\item radians in an arcsecond
\end{itemize}


\section{}

Compute teh angular separation of the Earth-Moon system as seen from the Sun The Earth-Moon distance is $D_M = 384,000\km$. $1\AU = \pow{1.5}{13}$


\section{}

The diameter of Jupiter is $143,000\km$. the closest the Earth and Jupiter ever get is about $4\AU$ or $\pow{6}{13}$.

\subsection{}

What is the angular diamter of Jupiter as seen from the Earth?

\subsection{}

Can you see this with your unaided eye? How about with a small telescope?


\section{}

The angular diameter of Saturn's ring system is $39 \arcsec$. The distance to Saturn is $\pow{1.3}{14}\cm$. What is the physical diameter of Saturn's rings?


\section{}

Venus is $0.28\AU$ from the Earth during a transit of the sun. Two observers separated by a North-South distance of $R_E = \pow{6.37}{8}\cm$ will see the path of Venus across the solar dish displaced by a parallax of $31.3\arcsec$. How big is $1\AU$?


\section{}

Pluto is $40\AU = \pow{6}{14}\cm$ from the Earth. The Pluto-Charon separation is $a = 19,600\km$. What is the Pluto-Charon angular separation as seen from the Earth? Is the Pluto-Charon system resolvable with a ground based telescope?


\section{}

The distance from Earth to Jupiter is $4\AU = \pow{6}{13}\cm$. The radii of the orbits of the 4 large moons of Jupiter (Galilean Satellites) are given below. Compute the angular sizes of the orbits as seen from Earth.

\begin{table}[H]
\centering
\begin{tabular}{c c}
\toprule
Moon & Distance to Jupiter (Moon to Jupiter) \\
\midrule
Io & $422,000\km$ \\
Europa & $671,000\km$ \\
Ganymede & $1,070,000\km$ \\
Callisto & $1,880,000\km$ \\
\bottomrule
\end{tabular}
\caption{Distances to Jupiter from Jupiter's moons}
\end{table}


\section{}

The parallax of the star Sirius is $0.37\arcsec$. What is the distance to Sirius  in (a) parsecs, (b) meters, and (c) light years. Note that the speed of light, $c$, is $c = \pow{3}{10}\cm/\s$


\section{}

The parallax of the closest star, $\alpha$Cen, is $0.75\arcsec$.

\subsection{}

What is the distance to $\alpha$Cen?

\subsection{}

What is the angular separation between the Sun and Jupiter as seen from $\alpha$Cen? The Sun-Jupiter distance is $5.2\AU$.

\end{document}
